% main.tex
% Compile with LuaLaTeX (recommended) or XeLaTeX
\documentclass[%
    colorscheme=default,   % Options: default, green, blue, red, purple
    fontsize=42pt,        % Options: 36pt, 42pt, 48pt, ...
    papersize=a0paper,    % Options: a0paper, a1paper, a2paper, a3paper, a4paper, letterpaper, ...
    orientation=portrait, % Options: portrait, landscape
    debug=false           % Options: true, false
]{betterposterv2}

% --- Poster Layout (optional overrides) ---
% Customize poster geometry (defaults shown):
% \SetHeaderHeight{20cm}
% \SetFooterHeight{10cm}
% \SetSideMargin{3cm}

% --- Color Customization (optional) ---
% Override individual colors (defaults use colorscheme option):
% \SetPageColor{bpLightGrey}
% \SetHeaderColor{bpDark}
% \SetFooterColor{bpDark}
% \SetLightBoxColor{bpWhite}
% \SetDarkBoxColor{bpMediumGrey}
% \SetHeaderTextColor{white}
% \SetFooterTextColor{white}

% --- Font Size Customization (optional) ---
% Customize font sizes (defaults shown):
% \SetMainFindingSize{\Huge}
% \SetTitleSize{\Large}
% \SetBoxTitleSize{\large}
% \SetSectionSize{\Large}

% --- Poster Content Configuration ---
% Your main finding - This is the most important message of your poster
\mainfinding{Teach people what you learned in 5 seconds (main takeaway/findings, not the title)}

% Your poster title
\title{Your Poster Title: Investigating the dynamics of X on Y}

% --- Footer Configuration ---
% QR code to your project/website (optional - comment out if not needed)
\qrcode{example-image}
% QR code label/subtitle (optional - appears below QR code)
\qrcodelabel{\href{https://example.org}{example.org}}  % Or a simple: Scan me!
% Authors - use \textsuperscript{} for affiliation markers if needed
\authors{First Author, Second Author}
% Affiliations - can use line breaks with \\
\affiliations{Department of Example Studies, University of Example}
% Institution logo (optional)
\logo{example-image}
% Publication year
\publicationyear{2026}
% Author portrait/photo (optional - comment out if not needed)
\portrait{example-image}

\begin{document}

\maketitle

% --- BLOCK 1: BACKGROUND (Wide Grey Box) ---
% Use bpcolor=grey for darker background, bpstyle=wide for full-width
% Use bptitle=transparent to make title background blend with page
\begin{bpbox}[bpcolor=grey, bpstyle=wide, title={Background}]
    Lorem ipsum dolor sit amet, consectetur adipiscing elit. Sed do eiusmod tempor incididunt ut labore et dolore magna aliqua. Ut enim ad minim veniam, quis nostrud exercitation ullamco laboris nisi ut aliquip ex ea commodo consequat. Duis aute irure dolor in reprehenderit in voluptate velit esse cillum dolore eu fugiat nulla pariatur.
\end{bpbox}

% --- BLOCK 2: TWO-COLUMN SECTION ---
% Section headers align with box titles
% \bpsection{Results}
% Use bpcolumns[2] for 2 columns, bpcolumns[3] for 3 columns, etc.
\begin{bpcolumns}[2]
    % Left column box
    \begin{bpbox}[title={Result 1}, bptitlebg=transparent]  % Titlebg can be made transparent
        Lorem ipsum dolor sit amet, consectetur adipiscing elit. Sed do eiusmod tempor incididunt ut labore et dolore magna aliqua.

        % Example of including a figure with caption
        \begin{center}
            \includegraphics[width=\linewidth, height=8cm, keepaspectratio]{example-image-a}%
            \captionof{figure}{Example figure caption describing the content}
        \end{center}
    \end{bpbox}
    % Right column box, make sure theres no empty line between boxes! 
    \begin{bpbox}[bptitlebg=transparent, title={Result 2}]  % Titlebg can be made transparent
        Ut enim ad minim veniam, quis nostrud exercitation ullamco laboris nisi ut aliquip ex ea commodo consequat.

        % Example of code block
        \textbf{Example Code:}
        \begin{verbatim}
def example_function(input_data):
    # Process the data
    result = input_data * 2
    return result
        \end{verbatim}
    \end{bpbox}
\end{bpcolumns}

% --- BLOCK 3: Methods ---
\bpsection{Methods}
% Standard white box (default)
\begin{bpbox}[title={}]  % Title can be omitted as well, which removes the title bar
    % Example of side-by-side content using minipages
    \begin{minipage}{0.45\textwidth}
        Lorem ipsum dolor sit amet, consectetur adipiscing elit. Sed do eiusmod tempor incididunt ut labore et dolore magna aliqua. Ut enim ad minim veniam, quis nostrud exercitation ullamco laboris.

        % Example of using siunitx for units
        The measured value was \SI{42.5}{\milli\meter} with an uncertainty of \SI{0.5}{\milli\meter}.
    \end{minipage}%
    \hfill
    \begin{minipage}{0.5\textwidth}
        \centering
        \includegraphics[width=\linewidth,height=6cm, keepaspectratio]{example-image-b}%
        \captionof{figure}{Results visualization}
    \end{minipage}
\end{bpbox}

% --- BLOCK 4: LIMITATIONS (Wide Grey Box) ---
% The title can be put outside the box, by making the title background transparent
\begin{bpbox}[bpcolor=grey, bpstyle=wide, title={Limitations}]
    Duis aute irure dolor in reprehenderit in voluptate velit esse cillum dolore eu fugiat nulla pariatur. Excepteur sint occaecat cupidatat non proident, sunt in culpa qui officia deserunt mollit anim id est laborum. Lorem ipsum dolor sit amet, consectetur adipiscing elit.
\end{bpbox}

\end{document}