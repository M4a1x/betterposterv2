% main.tex
% Compile with LuaLaTeX (recommended) or XeLaTeX
\documentclass{betterposterv2}

% --- Packages ---
\usepackage{lipsum} 
\usepackage{booktabs} % For nice tables
\usepackage{enumitem} % For custom lists

% --- Poster Content Configuration ---

% 1. The "Main Finding" (Top Header)
\mainfinding{A \$50 Open-Source NMR Spectrometer achieves Hz-level resolution using FPGA-based SDR and Python processing.}

% 2. The Main Title (Body)
\title{Democratizing Spin Physics: A Software-Defined Radio Approach to Low-Field NMR Spectroscopy}

% 3. The Footer
% We use minipages to organize the footer: [ QR Code | Authors | Logo ]
\footercontent{
    \begin{minipage}{0.15\textwidth}
        % Replace with your QR code image
        \includegraphics[height=8cm, keepaspectratio]{example-image-1x1}
    \end{minipage}%
    \hfill
    \begin{minipage}{0.6\textwidth}
        \Large
        \textbf{Authors:}\\
        Your Name$^{1}$, Supervisor Name$^{2}$\\
        $^{1}$Department of Electrical Engineering, University of Schifferstadt\\
        $^{2}$Institute for Chemical Technology
    \end{minipage}%
    \hfill
    \begin{minipage}{0.2\textwidth}
        \raggedleft
        % Replace with your logo
        \includegraphics[height=6cm, keepaspectratio]{example-image-16x9}
    \end{minipage}
}

\begin{document}

% Print the Title
\maketitle

% --- BLOCK 1: BACKGROUND (Wide Grey) ---
\begin{bpwidebox}{Background}
    Current commercial NMR spectrometers cost upwards of \$50k, limiting access in education and developing regions. We propose a \textbf{25 MHz (0.58 T)} permanent magnet system using commercial-off-the-shelf (COTS) RF components and a custom open-source Python analysis stack to democratize magnetic resonance.
\end{bpwidebox}

% --- BLOCK 2: METHODS & SYSTEM (Two Columns) ---
\bpsection{System Design \& Methods}

% The 'bpcolumns' environment automatically places boxes side-by-side
\begin{bpcolumns}[2]

    % --- LEFT BOX: Hardware ---
    \begin{bpwhitebox}[title={Hardware Architecture}]
        The system relies on a Red Pitaya FPGA board acting as a Software Defined Radio (SDR).

        \begin{itemize}[leftmargin=*]
            \item \textbf{Magnet:} Halbach array of N52 NdFeB magnets ($B_0 \approx 0.58$\,T).
            \item \textbf{Probe:} Solenoid coil ($L=1.2\,\mu$H) with capacitive matching to $50\,\Omega$.
            \item \textbf{Transmitter:} 20W Class-E Power Amplifier.
            \item \textbf{Receiver:} Low Noise Amplifier (LNA) with active T/R switch.
        \end{itemize}

        \begin{center}
            % Placeholder image
            \includegraphics[width=\linewidth, height=8cm, keepaspectratio]{example-image-a}
            \captionof{figure}{RF Signal Chain Block Diagram}
        \end{center}
    \end{bpwhitebox}

    % --- RIGHT BOX: Software ---
    \begin{bpwhitebox}[title={Python Processing Stack}]
        Data acquisition and processing are handled purely in Python, utilizing \texttt{numpy} and \texttt{scipy}.

        \textbf{Pulse Sequence:}
        A standard CPMG sequence is generated digitally.

        \textbf{Processing Pipeline:}
        \begin{enumerate}[leftmargin=*]
            \item Digital Down Conversion (DDC) from 25 MHz to Baseband.
            \item CIC Decimation filters.
            \item Phase correction (0th and 1st order).
            \item FFT and Voigt Profile fitting.
        \end{enumerate}

        % Example code snippet
        \begin{verbatim}
def process_signal(raw_data):
    # Downconvert
    iq = raw_data * np.exp(-1j * w_c * t)
    # Decimate
    return scipy.signal.decimate(iq, 10)
        \end{verbatim}
    \end{bpwhitebox}

\end{bpcolumns}

% --- BLOCK 3: RESULTS (Full Width, but split internally) ---
\bpsection{Results}

\begin{bpwhitebox}
    \begin{minipage}{0.45\textwidth}
        \textbf{Ethanol Spectrum Analysis:}\\
        The figure on the right demonstrates a single-shot acquisition of an Ethanol sample.

        We observe clearly resolved J-coupling in the methyl triplet and methylene quartet. The linewidth (FWHM) is approximately \textbf{4.5 Hz} after shimming.

        \vspace{1em}
        \begin{tabular}{ll}
            \toprule
            \textbf{Parameter} & \textbf{Value} \\
            \midrule
            Larmor Freq.       & 24.95 MHz      \\
            SNR (Single)       & $> 500$        \\
            \bottomrule
        \end{tabular}
    \end{minipage}%
    \hfill
    \begin{minipage}{0.5\textwidth}
        \centering
        \includegraphics[width=\linewidth, height=12cm, keepaspectratio]{example-image-b}
        \captionof{figure}{25 MHz 1H-NMR spectrum of Ethanol.}
    \end{minipage}
\end{bpwhitebox}

% --- BLOCK 4: LIMITATION (Wide Grey) ---
\begin{bpwidebox}{Limitation}
    Field homogeneity remains the primary challenge. Thermal drift of the permanent magnet shifts the Larmor frequency by approx $200$\,Hz/$^\circ$C, requiring an external $B_0$ lock loop or temperature stabilization chamber for long-term averaging.
\end{bpwidebox}

\end{document}