% main.tex
% Compile with LuaLaTeX (recommended) or XeLaTeX
\documentclass{betterposterv2}

% --- Packages ---
\usepackage{booktabs}
\usepackage{enumitem}

% --- Poster Layout (optional overrides) ---
% \SetHeaderHeight{14cm}
% \SetFooterHeight{12cm}
% \SetSideMargin{4.5cm}
% \SetColumnGap{1.2cm}

% --- Poster Content Configuration ---
\mainfinding{A \$50 Open-Source NMR Spectrometer achieves \unit{\hertz}-level resolution using FPGA-based SDR and Python processing.}

\title{Democratizing Spin Physics: A Software-Defined Radio Approach to Low-Field NMR Spectroscopy}

% --- Footer Configuration ---
\qrcode{example-image-1x1}
\authors{Your Name\textsuperscript{1}, Supervisor Name\textsuperscript{2}}
\affiliations{%
    \textsuperscript{1}Department of Electrical Engineering, University of Schifferstadt\\\\
    \textsuperscript{2}Institute for Chemical Technology%
}
\logo{example-image-16x9}

\begin{document}

\maketitle

% --- BLOCK 1: BACKGROUND (Wide Grey) ---
\begin{bpbox}[bpcolor=grey, bpstyle=wide, title={Background}]
    Current commercial NMR spectrometers cost upwards of \$50k, limiting access in education and developing regions. We propose a \SI{25}{\MHz} (\SI{0.58}{\tesla}) permanent magnet system using commercial-off-the-shelf (COTS) RF components and a custom open-source Python analysis stack to democratize magnetic resonance.
\end{bpbox}

% --- BLOCK 2: METHODS & SYSTEM (Two Columns) ---
\bpsection{System Design \& Methods}

\begin{bpcolumns}[2]

    \begin{bpbox}[title={Hardware Architecture}]
        The system relies on a Red Pitaya FPGA board acting as a Software Defined Radio (SDR).

        \begin{itemize}[leftmargin=*]
            \item \textbf{Magnet:} Halbach array of N52 NdFeB magnets ($B_0 \approx \SI{0.58}{\tesla}$).
            \item \textbf{Probe:} Solenoid coil ($L=\SI{1.2}{\micro\henry}$) with capacitive matching to \SI{50}{\ohm}.
            \item \textbf{Transmitter:} \SI{20}{\watt} Class-E Power Amplifier.
            \item \textbf{Receiver:} Low Noise Amplifier (LNA) with active T/R switch.
        \end{itemize}

        \begin{center}
            \includegraphics[width=\linewidth, height=8cm, keepaspectratio]{example-image-a}%
            \captionof{figure}{RF Signal Chain Block Diagram}
        \end{center}
    \end{bpbox}

    \begin{bpbox}[title={Python Processing Stack}]
        Data acquisition and processing are handled purely in Python, utilizing \texttt{numpy} and \texttt{scipy}.

        \textbf{Pulse Sequence:}
        A standard CPMG sequence is generated digitally.

        \textbf{Processing Pipeline:}
        \begin{enumerate}[leftmargin=*]
            \item Digital Down Conversion (DDC) from \SI{25}{\MHz} to Baseband.
            \item CIC Decimation filters.
            \item Phase correction (0th and 1st order).
            \item FFT and Voigt Profile fitting.
        \end{enumerate}

        \begin{verbatim}
def process_signal(raw_data):
    # Downconvert
    iq = raw_data * np.exp(-1j * w_c * t)
    # Decimate
    return scipy.signal.decimate(iq, 10)
        \end{verbatim}
    \end{bpbox}

\end{bpcolumns}

% --- BLOCK 3: RESULTS ---
\bpsection{Results}

\begin{bpbox}[title={Results Overview}]
    \begin{minipage}{0.45\textwidth}
        \textbf{Ethanol Spectrum Analysis:}\\
        The figure on the right demonstrates a single-shot acquisition of an Ethanol sample.

        We observe clearly resolved J-coupling in the methyl triplet and methylene quartet. The linewidth (FWHM) is approximately \textbf{\SI{4.5}{\hertz}} after shimming.

        \vspace{1em}
        \begin{tabular}{ll}
            \toprule
            \textbf{Parameter} & \textbf{Value}   \\
            \midrule
            Larmor Freq.       & \SI{24.95}{\MHz} \\
            SNR (Single)       & $> 500$          \\
            \bottomrule
        \end{tabular}
    \end{minipage}%
    \hfill
    \begin{minipage}{0.5\textwidth}
        \centering
        \includegraphics[width=\linewidth, height=12cm, keepaspectratio]{example-image-b}%
        \captionof{figure}{\SI{25}{\MHz} \textsuperscript{1}H-NMR spectrum of Ethanol.}
    \end{minipage}
\end{bpbox}

% --- BLOCK 4: LIMITATION (Wide Grey) ---
\begin{bpbox}[bpcolor=grey, bpstyle=wide, title={Limitation}]
    Field homogeneity remains the primary challenge. Thermal drift of the permanent magnet shifts the Larmor frequency by approx \SI{200}{\hertz\per\degreeCelsius}, requiring an external $B_0$ lock loop or temperature stabilization chamber for long-term averaging.
\end{bpbox}

\end{document}